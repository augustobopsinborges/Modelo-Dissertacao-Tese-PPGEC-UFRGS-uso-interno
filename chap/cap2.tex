% %%%%%%%%%%%%%%%%%%%%%%%%%%%%%%%%%%%%%%%%%%%%%%%%%%%%%%%%%%%%%%%%%%%%%%%%%%%%%%%%%
%
%	cap#.tex 
%  -----------
%
%	Neste arquivo você escreve o capítulo. Deve sempre iniciar com o ambiente
%	\chapter{Nome do Capitulo}
%
% %%%%%%%%%%%%%%%%%%%%%%%%%%%%%%%%%%%%%%%%%%%%%%%%%%%%%%%%%%%%%%%%%%%%%%%%%%%%%%%%%
%
\chapter{Iniciando com o \LaTeX}

Nesse capítulo se dará um breve histórico e introdução sobre o que é o \TeX~e \LaTeX.

% ---
\section{O que é \TeX?}
% ---

O \TeX~é um programa de processamento de texto criado por Donald E. Knuth \cite{Knuth1984} a pedido da \textbf{AMS}\footnote{American Mathematical Society} orientado à composição, impressão de textos e fórmulas matemáticas. Esse programa interpreta cerca de 600 comandos que controlam a construção do layout de uma página (fonte de letra a usar, espaçamento entre linhas, organização de equações, entre outros aspectos referente ao texto final impresso). Para as fontes, Knuth aproveitou a experiência de antigos tipógrafos e desenvolveu o programa \texttt{METAFONT} para criá-las. Por isso, às vezes, percebe-se uma incrível semelhança entre livros antigos e os tipos de fontes utilizados pelo \TeX.

Pode se considerar o \TeX~ como sendo um compilador para textos científicos de alta qualidade. Contudo, sua aprendizagem e utilização não é tão fácil para qualquer usuário de computador. Felizmente, quase que simultaneamente foi desenvolvido o \LaTeX para simplificar a utilização do \TeX.

O desenvolvimento do \TeX~ se estendeu de 1977 a 1986, quando o mesmo foi posto de forma gratuita para ser utilizado. Atualmente o \TeX~ e o \texttt{METAFONT} não estão mais em desenvolvimentos e, segundo seu autor\footnote{Donald E. Kenuth. The Future of \TeX and Metafont. TUGboat, 11(4):489, novembro de 1990}, não serão realizados mais mudanças futuras, exceto correções de erros na programação.

% ---
\section{O que é \LaTeX?}
% ---

O \LaTeX, é um programa desenvolvido por Leslie Lamport \cite{1994lamport} que consiste em interpretar um conjunto de macros (que são instruções ou comandos) para simplificar o uso da linguagem \TeX. A utilização desses comandos permitem o autor compor e configurar seu documento de modo mais simples utilizando estruturas pré definidas.

Ao contrário do \TeX~ o desenvolvimento do \LaTeX~é crescente e possui diversos pacotes adicionais para realizar uma imensa quantidade de tarefas diferentes no processamento de textos.

O arquivo de entrada do \LaTeX~ está no formato ASCII e pode ser criado com qualquer editor de texto, como por exemplo o Bloco de Notas do Windows. Esse arquivo possuí tanto o texto que será impresso quanto as instruções que o \LaTeX~ interpretará para compor o texto no layout da página. Um projeto em \LaTeX~ pode conter diversos arquivos de entrada.

Existem vários editores de \LaTeX~ que servem de ambiente para escrever os arquivos de entrada. Esses programas possuem corretor ortográfico interativo, análise lexical (contagem de palavras e frases), realce de sintaxe, chamam e configuram o compilador, apresentam janelas de aviso e erros de compilação, permitem visualizar o documento impresso e possuem acesso rápido a comandos \LaTeX~ para fazer tabelas, fórmulas e etc. São exemplos de editores o TeXstudio, LyX, TeXmaker e OverLeaf. Para comparações entre diversos editores de \LaTeX~ veja \citeonline{wiki:2020}.

% ---
\section{Diferença entre processadores de texto Visuais e Lógicos}
% ---

Os programas que processam textos podem ser dividido em duas categorias:

\begin{alineas}
	
	\item Visuais (WYSIWYG): o texto que você digita aparece na tela da mesma forma que vai ser impresso. Isso é conhecido como WHAT YOU SEE IS WHAT YOU GET. Um exemplo é o Microsoft Word; 

	\item Lógico: o texto é digitado em um arquivo fonte junto com instruções de compilação. O arquivo é então compilado e gera uma saída que pode ser visualizada. Exemplo: HTML e \LaTeX.
	
\end{alineas}

Os processadores Visuais são bastante intuitivos e produzem documentos esteticamente bonitos. Porém pode ocorrer corrupção do arquivo, incompatibilidade na leitura entre versões diferentes do programa e problemas de configuração da formatação, uma vez que esta depende de diversos parâmetros do programa que podem variar sem muito controle do usuário.

Os processadores Lógicos, apesar de serem menos intuitivos para um usuário acostumado com o sistema Visual, possuem menos chances de corromper o arquivo e maior estabilidade na formatação, uma vez que é necessário indicar a estrutura lógica da mesma no texto. Além disso, pode-se citar algumas vantagens do \LaTeX:

\begin{alineas}
	
	\item é altamente portável e grátis. O sistema funciona em qualquer plataforma computacional;
		
	\item existem pacotes adicionais sem custo algum para muitas tarefas tipográficas que não estão inclusa no pacote do \LaTeX~básico;
	
	\item o usuário só precisa introduzir instruções simples para indicar a estrutura do documento e sua formatação. Os textos ficam bem estruturados no arquivo de entrada;

	\item estruturas como notas de rodapé, bibliografias, índices, tabelas e outras podem ser produzidas sem grande esforço;
	
	\item existem comunidades de dúvidas e desenvolvimento de pacotes por toda a internet. Uma dúvida que você tenha, certamente alguém já teve;
	
	\item é possível criar um conjunto de macros ou mesmo utilizar e customizar pacotes que contém uma série de macros criadas pela comunidade para configuração da formatação, como a classe Abn\TeX2 e o pacote abntex2cite;
	
\end{alineas}

Como uma das desvantagens pode-se citar que produzir um design específico de um documento que não esteja entre os pré-definidos pode ser uma tarefa de criação que leva tempo.

A mecânica do \LaTeX~ é aquela em que o autor escreve os arquivos de entrada *.tex (geralmente em um editor de \LaTeX) contendo o texto e os comandos de formatação na linguagem \LaTeX~ e então executa o compilador (um programa) que, seguindo as regras dessa linguagem, cria um arquivo de impressão formatado (*.pdf, *.dvi) pronto para ser visualizado.

% ---
\section{Instalando \LaTeX~ no sistema Windows}
% ---

Para utilizar o \LaTeX~no Windows você precisa primeiramente instalar o MiKTeX. O MiKTeX é uma distribuição multiplataforma que contém o compilador do \LaTeX, alguns pacotes básicos e outras ferramentas auxiliares (como um gerenciador de pacotes e um editor de \LaTeX~ chamado TeXworks). Posteriormente, é aconselhado instalar um editor de \LaTeX~ com mais funcionalidades, como o TeXstudio.

Para instalar o MiKTeX você deverá:

\begin{alineas}
	
	\item baixar o instalador do MiKTeX básico no site \url{https://miktex.org/download} correspondente ao seu sistema Windows 64-bit ou 32-bit;
	
	\item rode o instalador com privilégio de administrador;
	
	\item selecione instalar para todos os usuários e escolha a pasta. Recomenda-se manter o local sugerido;
	
	\item escolha as opções solicitadas, recomenda-se manter sugestão de folha \textbf{A4} para o tipo de papel e \textbf{Ask me First} para instalar os pacotes faltantes;
	
	\item clique em \textbf{start} para iniciar a instalação;
	
	\item após terminar feche o instalador;
	
\end{alineas}

Para instalar o TeXstudio você deverá:

\begin{alineas}
	
	\item baixar o instalador do TeXstudio no site \url{https://www.texstudio.org/} correspondente ao seu sistema Windows 64-bit ou 32-bit;
	
	\item rode o instalador com privilégio de administrador e escolha a língua de preferência;
	
	\item escolha a pasta de instalação. Recomenda-se manter o local sugerido;
	
	\item selecione criar ícones no desktop e na barra de acesso rápido;
	
	\item clique em \textbf{avançar} para iniciar a instalação;
	
	\item após terminar feche o instalador;
	
\end{alineas}

% ---
\section{Instalando \LaTeX~ em outros sistemas operacionais}
% ---

A instalação do \LaTeX~ em outros sistemas (Linux e Mac) segue passos equivalentes ao da seção anterior. Tanto o TeXstudio quanto o MikTeX possuem versões para essas outras plataformas. Contudo, verifique se todos os pacotes requeridos pelo arquivo ppgec.cls estão baixados. Caso não estejam, entre no MikTeX e baixe-os.

% ---
\section{Alguns tutoriais úteis}
% ---

Há muitos tutorias sobre \LaTeX~ disponíveis na internet. Abaixo segue a lista de alguns links que você pode achar útil:

\begin{alineas}
	
	\item curso básico sobre LaTeX no TeXstudio:\\ \url{https://youtube.com/playlist?list=PLTCY9jGgE91dFJwOP4wPvwLdyNj2Bch0D};
	
	\item curso básico ao avançado sobre OverLeaf:\\ \url{https://youtube.com/playlist?list=PLF6ZF9NW0WmqUAgtkYlQmCDHP6H_bYwGk};
	
	\item slides de um curso básico de \LaTeX~ do Alcemir Rodrigues Santos da UFBA:\\ \url{https://pt.slideshare.net/alcemirsantos/erbase-2015-curso-bsico-de-latex}
	
	\item site para formatar tabela e gerar código em \LaTeX:\\ \url{https://www.tablesgenerator.com/};
	
	\item site para gerar equações em \LaTeX:\\  \url{https://www.codecogs.com/latex/eqneditor.php?lang=pt-br};
	
	\item como obter uma bibliografia e colocar no seu trabalho:\\ \url{http://XXXXX};
	
\end{alineas}


% ---
\section{Sobre o modelo PPGEC, organização dos arquivos e utilização}
% ---
\begin{alineas}

	\item para baixar o modelo PPGEC: \\ \url{http://www.XXXXX}.

	\item organização dos arquivos e utilização do modelo PPGEC:\\ \url{https://XXXXX};

	\item modelo PPGEC com o Overleaf:\\ \url{https://www.youtube.com/watch?v=wK4mSMzB6KQ&t=184s};
	
\end{alineas}


