% ------------------------------------------------------------------------
% ------------------------------------------------------------------------
% Elaboração das modificações para uso interno do PPGEC-UFRGS
% Augusto Bopsin Borges (augusto.borges@ufrgs.br)
% Versão: 05/12/2019
% ------------------------------------------------------------------------
% ------------------------------------------------------------------------


\documentclass[
	% -- opções da classe memoir --
	12pt,				% tamanho da fonte
	openright,			% capítulos começam em pág ímpar (insere página vazia caso preciso)
	oneside,			% para impressão em em apenas um lado. Oposto a twoside (impressão em frente e verso)
	a4paper,			% tamanho do papel.
	% -- opções da classe abntex2 --
	chapter=TITLE,		% títulos de capítulos convertidos em letras maiúsculas
	section=TITLE,		% títulos de seções convertidos em letras maiúsculas
	% -- opções do pacote babel --
	english,			% idioma adicional para hifenização
	french,				% idioma adicional para hifenização
	spanish,			% idioma adicional para hifenização
	brazil				% o último idioma é o principal do documento
	]{ppgec}
% ---
% Pacotes básicos
% ---
\usepackage{times}		  	    % Usa a fonte Times apenas para texto - usar:
\usepackage{mathptmx}         % para times tbm nas equações
\usepackage[T1,LGRx,T1]{fontenc}  		% Selecao de codigos de fonte.
\usepackage[utf8]{inputenc}		% Codificacao do documento (conversão automática dos acentos)
\usepackage{color}			    	% Controle das cores
\usepackage{graphicx}		    	% Inclusão de gráficos
\graphicspath{{./figuras/}}   % Caminho para a pasta que contém os arquivos das Figuras
\usepackage{microtype} 		  	% para melhorias de justificação
\usepackage{fancyhdr}
\usepackage{amssymb,amsmath,amsfonts,textcomp,bm}
\usepackage{pdfpages}
% ---

% ---
% Pacotes adicionais, usados apenas no âmbito do Modelo Canônico do abnteX2
% ---
\usepackage{lipsum}				% para geração de dummy text
% ---

% ---
% Pacotes de citações
% ---
\usepackage[brazilian,hyperpageref]{backref}	 % Paginas com as citações na bibl
\usepackage[alf]{abntex2cite}	% Citações padrão ABNT


% ---
% Incluído por Augusto B. Borges
% ---
\usepackage{caption}
% Tabelas
\usepackage{array}								% Elementos extras para formatação de tabelas
\usepackage{booktabs}							% Tabelas com qualidade de publicação
\usepackage{longtable}							% Para criar tabelas maiores que uma página
\usepackage{lscape}								% adicionar tabelas e figuras como landscape
% Notas de rodapé
\usepackage{footnote}							% Lidar com notas de rodapé em diversas situações
\makesavenoteenv{tabular}						% Notas criadas nas tabelas ficam no fim das tabelas

% ---
% CONFIGURAÇÕES DE PACOTES
% ---

% ---
% Configurações do pacote backref
% Usado sem a opção hyperpageref de backref
\renewcommand{\backrefpagesname}{Citado na(s) página(s):~}
% Texto padrão antes do número das páginas
\renewcommand{\backref}{}
% Define os textos da citação
\renewcommand*{\backrefalt}[4]{
	\ifcase #1 %
		Nenhuma citação no texto.%
	\or
		Citado na página #2.%
	\else
		Citado #1 vezes nas páginas #2.%
	\fi}%
% ---         % Pacotes e comandos customizados

% ------------------------------------------------------------------------
% ------------------------------------------------------------------------
% ----------------------------------------------------------
% Informações do Aluno e outros dados 
% ----------------------------------------------------------

\titulo{Título Completo da Dissertação ou Tese}
\autor{Fulano de Tal}
\newcommand{\citarautor}{Tal, F. D.}
\newcommand{\emailautor}{fulano.de.tal@ufrgs.br}
\local{Porto Alegre}
\data{201X}
\newcommand{\datacompleta}{[dia] de [mês] de 201X}
\orientador{Nome do Orientador}
\newcommand{\titorientador}{Dr. pela Universidade de Origem}
%\coorientador{Equipe \abnTeX}
\instituicao{Universidade Federal do Rio Grande do Sul}
\newcommand{\abrevinstituicao}{UFRGS}
\newcommand{\escola}{Escola de Engenharia}
\newcommand{\ppg}{Programa de Pós-Graduação em Engenharia Civil}
\newcommand{\abrevppg}{PPGEC}
\tipotrabalho{Tese de Doutorado}
\newcommand{\tipotrabalhocurto}{Tese}
\newcommand{\cursorealizado}{Doutorado em Engenharia}
\newcommand{\tituloobtido}{Doutor em Engenharia}
\newcommand{\areaconcentracao}{Geotecnia}
% O preambulo deve conter o tipo do trabalho, o objetivo,
% o nome da instituição e a área de concentração
\preambulo{\imprimirtipotrabalho~apresentada ao \ppg~da \imprimirinstituicao~como parte dos requisitos para obtenção do título de \tituloobtido.}

% Dados do coordenador do curso
\newcommand{\coordenador}{Nilo Consoli}
\newcommand{\titcoordenador}{Ph.D. pela Concordia University, Canadá}

% Membros da Banca e titulação
\newcommand{\tratbancaum}{Prof.}
\newcommand{\bancaum}{Luís Carlos Prestes}
\newcommand{\origembancaum}{UFRGS}
\newcommand{\titbancaum}{Ph.D. pela Universidade de Origem, País}

\newcommand{\tratbancadois}{Prof.}
\newcommand{\bancadois}{Leonel de Moura Brizola}
\newcommand{\origembancadois}{UFRGS}
\newcommand{\titbancadois}{Dr. pela Universidade Federal do Rio Grande do Sul}

\newcommand{\tratbancatres}{Prof.}
\newcommand{\bancatres}{Getúlio Vargas}
\newcommand{\origembancatres}{UFRGS}
\newcommand{\titbancatres}{Dr. pela Universidade Federal do Rio Grande do Sul}

\newcommand{\tratbancaquatro}{Profa.}
\newcommand{\bancaquatro}{Anita Garibaldi}
\newcommand{\origembancaquatro}{UFSC}
\newcommand{\titbancaquatro}{Dra. pela Universidade Federal de Santa Catarina}
% ---


% NÃO MEXER DAQUI PARA BAIXO

% ---
% Configurações de aparência do PDF final

% alterando o aspecto da cor azul
\definecolor{blue}{RGB}{41,5,195}

% informações do PDF
\makeatletter
\hypersetup{
     	%pagebackref=true,
		pdftitle={\@title},
		pdfauthor={\@author},
    	pdfsubject={\imprimirpreambulo},
	    pdfcreator={LaTeX with abnTeX2},
		pdfkeywords={abnt}{latex}{abntex}{abntex2}{trabalho acadêmico},
		colorlinks=true,       		% false: boxed links; true: colored links
    	linkcolor=black,       	% color of internal links
    	citecolor=black,     		% color of links to bibliography
    	filecolor=black,     		% color of file links
		urlcolor=black,
		bookmarksdepth=4
}
\makeatother
% ---

% ---
% Posiciona figuras e tabelas no topo da página quando adicionadas sozinhas
% em um página em branco. Ver https://github.com/abntex/abntex2/issues/170
\makeatletter
\setlength{\@fptop}{5pt} % Set distance from top of page to first float
\makeatother
% ---

% ---
% Possibilita criação de Quadros e Lista de quadros.
% Ver https://github.com/abntex/abntex2/issues/176
%
\newcommand{\quadroname}{Quadro}
\newcommand{\listofquadrosname}{Lista de quadros}

\newfloat[chapter]{quadro}{loq}{\quadroname}
\newlistof{listofquadros}{loq}{\listofquadrosname}
\newlistentry{quadro}{loq}{0}

% configurações para atender às regras da ABNT
\setfloatadjustment{quadro}{\centering}
\counterwithout{quadro}{chapter}
\renewcommand{\cftquadroname}{\quadroname\space}
\renewcommand*{\cftquadroaftersnum}{\hfill--\hfill}

\setfloatlocations{quadro}{hbtp} % Ver https://github.com/abntex/abntex2/issues/176
% ---

% ---
% Espaçamentos entre linhas e parágrafos
% ---
\OnehalfSpacing % Definição do PPGEC-UFRGS
% O tamanho do parágrafo é dado por:
\setlength{\parindent}{0cm} % Definição do PPGEC-UFRGS

% Controle do espaçamento entre um parágrafo e outro:
\setlength{\parskip}{0.42336cm} % este é o espaçamento no modelo PPGEC, equivalente a 12pt (1pt = 0,03528cm)
% tente também \onelineskip

% ---
% compila o indice
% ---
\makeindex
% ---

% ----
% Início do documento
% ----
   % PREENCHA NESTE ARQUIVO SEUS DADOS (dadosdeentrada.tex)
% ------------------------------------------------------------------------
% ------------------------------------------------------------------------

\begin{document}
\frenchspacing             % Retira espaço extra obsoleto entre as frases.
% Seleciona o idioma do documento (conforme pacotes do babel)
% \selectlanguage{english}
\selectlanguage{brazil}
% ----------------------------------------------------------
% ELEMENTOS PRÉ-TEXTUAIS
% ----------------------------------------------------------

% ---
% Capa
% ---
\imprimircapa
% ---

% ---
% Folha de rosto
% (o * indica que haverá a ficha bibliográfica)
% ---
\imprimirfolhaderosto%*
% ---

% ---
% Inserir a ficha bibliografica
% ---

% Isto é um exemplo de Ficha Catalográfica, ou ``Dados internacionais de
% catalogação-na-publicação''. Você pode utilizar este modelo como referência.
% Porém, provavelmente a biblioteca da sua universidade lhe fornecerá um PDF
% com a ficha catalográfica definitiva após a defesa do trabalho. Quando estiver
% com o documento, salve-o como PDF no diretório do seu projeto e substitua todo
% o conteúdo de implementação deste arquivo pelo comando abaixo:
%

% --------------------
% \begin{fichacatalografica}
%     \includepdf{fig_ficha_catalografica.pdf}
% \end{fichacatalografica}

%\begin{fichacatalografica}
%	\ttfamily
%	\vspace*{\fill}					% Posição vertical
%	\begin{center}					% Minipage Centralizado
	% \fbox{\begin{minipage}[c][8cm]{13.5cm}		% Largura
	% \small
	% \imprimirautor
	%Sobrenome, Nome do autor

	% \hspace{0.5cm} \imprimirtitulo  / \imprimirautor. --
	% \imprimirlocal, \imprimirdata.

	% \thelastpage p.\\

% 	\hspace{0.5cm} \imprimirorientadorRotulo~\imprimirorientador\\
%
% 	\hspace{0.5cm}
% 	\parbox[t]{\textwidth}{\imprimirtipotrabalho~--~\imprimirinstituicao,
% 	\imprimirdata.}\\
%
% 	\hspace{0.5cm}
% 		1. Palavra-chave1.
% 		2. Palavra-chave2.
% 		2. Palavra-chave3.
% 		I. Sobrenome, Nome, orientador.
% 		II. Universidade xxx.
% 		III. Faculdade de xxx.
% 		IV. Título
% 	\end{minipage}}
% 	\end{center}
% \end{fichacatalografica}
% ---
% --------------------

% ---
% Inserir errata
% ---
% \begin{errata}
% Elemento opcional da \citeonline[4.2.1.2]{NBR14724:2011}. Exemplo:

% \vspace{\onelineskip}

% FERRIGNO, C. R. A. \textbf{Tratamento de neoplasias ósseas apendiculares com
% reimplantação de enxerto ósseo autólogo autoclavado associado ao plasma
% rico em plaquetas}: estudo crítico na cirurgia de preservação de membro em
% cães. 2011. 128 f. Tese (Livre-Docência) - Faculdade de Medicina Veterinária e
% Zootecnia, Universidade de São Paulo, São Paulo, 2011.

% \begin{table}[htb]
% \center
% \footnotesize
% \begin{tabular}{|p{1.4cm}|p{1cm}|p{3cm}|p{3cm}|}
%   \hline
%   \textbf{Folha} & \textbf{Linha}  & \textbf{Onde se lê}  & \textbf{Leia-se}  \\
    % \hline
    % 1 & 10 & auto-conclavo & autoconclavo\\
%   \hline
% \end{tabular}
% \end{table}

% \end{errata}
% ---

\begin{folhadeaprovacao}

  \begin{center}
    {\ABNTEXchapterfont\large\MakeUppercase{\imprimirautor}}

    \vspace*{\fill}
    \begin{center}
      \OnehalfSpacing{\ABNTEXchapterfont\bfseries\Large\MakeUppercase{\imprimirtitulo}}
    \end{center}
    \vspace*{\fill}
    \OnehalfSpacing{Esta \MakeTextLowercase{\imprimirtipotrabalho} foi julgada adequada para a obtenção do título de \MakeUppercase{\tituloobtido}, na área de concentração \areaconcentracao, e aprovada em sua forma final pelo professor orientador e pelo Programa de Pós-Graduação em Engenharia Civil da Universidade Federal do Rio Grande do Sul.}
    % melhorar o espaçamento entre linhas dessa parte
    \vspace*{36pt}
   \end{center}
    \begin{center}
    \imprimirlocal, \datacompleta
    \end{center}

    \begin{flushright}

    Prof. \imprimirorientador \\ \titorientador \\ Orientador
    \vspace{7mm}
    \\ Prof. \coordenador \\ \titcoordenador \\ Coordenador do \abrevppg/\abrevinstituicao
    \vspace{5mm}
    \\ \textbf{\MakeUppercase{Banca Examinadora}}
    \vspace{5mm}
    \\ \textbf{\tratbancaum~\bancaum~(\origembancaum)} \\ \titbancaum
    \vspace{5mm}
    \\ \textbf{\tratbancadois~\bancadois~(\origembancadois)} \\ \titbancadois
    \vspace{5mm}
    \\ \textbf{\tratbancatres~\bancatres~(\origembancatres)} \\ \titbancatres
    \vspace{5mm}
    \\ \textbf{\tratbancaquatro~\bancaquatro~(\origembancaquatro)} \\ \titbancaquatro

    \end{flushright}

    \vspace*{\fill}
    \vspace*{\fill}


  \end{folhadeaprovacao}
%   ---              % Configura elementos pré-textuais

% ------------------------------------------------------------------------
% ------------------------------------------------------------------------
% ----------------------------------------------------------
% Neste arquivo edita-se a DEDICATÓRIA, os AGRADECIMENTOS, a EPÍGRAFE, os RESUMOS, a LISTA DE ABREVIATURAS E SIGLAS e a LISTA DE SÍMBOLOS
% ----------------------------------------------------------

% ---
% Dedicatória
% ---
\begin{dedicatoria}
   \vspace*{\fill}
	\begin{flushright}
		Este trabalho é dedicado às crianças adultas que \\
		quando pequenas, sonharam em se tornar cientistas.
	\end{flushright}
\end{dedicatoria}
% ---

% ---
% Agradecimentos
% ---
\begin{agradecimentos}
Os agradecimentos principais são direcionados à Gerald Weber, Miguel Frasson,
Leslie H. Watter, Bruno Parente Lima, Flávio de Vasconcellos Corrêa, Otavio Real
Salvador, Renato Machnievscz\footnote{Os nomes dos integrantes do primeiro
projeto abn\TeX\ foram extraídos de
\url{http://codigolivre.org.br/projects/abntex/}} e todos aqueles que
contribuíram para que a produção de trabalhos acadêmicos conforme
as normas ABNT com \LaTeX\ fosse possível.

Agradecimentos especiais são direcionados ao Centro de Pesquisa em Arquitetura
da Informação\footnote{\url{http://www.cpai.unb.br/}} da Universidade de
Brasília (CPAI), ao grupo de usuários
\emph{latex-br}\footnote{\url{http://groups.google.com/group/latex-br}} e aos
novos voluntários do grupo
\emph{\abnTeX}\footnote{\url{http://groups.google.com/group/abntex2} e
\url{http://www.abntex.net.br/}}~que contribuíram e que ainda
contribuirão para a evolução do \abnTeX.

\end{agradecimentos}
% ---

% ---
% Epígrafe
% ---
\begin{epigrafe}
    \vspace*{\fill}
	\begin{flushright}
		A gente quer passar um rio a nado, e passa; mas vai dar na \\
		outra banda é num ponto muito mais abaixo, bem diverso \\
		do em que primeiro se pensou. Viver nem não é muito \\
		perigoso? \\
		\textit{Grande Sertão: Veredas -- Guimarães Rosa}
	\end{flushright}
\end{epigrafe}
% ---

% ---
% RESUMOS
% ---

% resumo em português
\setlength{\absparsep}{18pt} % ajusta o espaçamento dos parágrafos do resumo
\begin{resumo}

 \SingleSpacing{\citarautor~{\bfseries \rmfamily \fontsize{12}{12} \imprimirtitulo}. \imprimirdata. \thelastpage p. \tipotrabalhocurto~(\cursorealizado) -- \ppg, \imprimirinstituicao, \imprimirlocal.}

 Segundo a \citeonline[3.1-3.2]{NBR6028:2003}, o resumo deve ressaltar o
 objetivo, o método, os resultados e as conclusões do documento. A ordem e a extensão
 destes itens dependem do tipo de resumo (informativo ou indicativo) e do
 tratamento que cada item recebe no documento original. O resumo deve ser
 precedido da referência do documento, com exceção do resumo inserido no
 próprio documento. (\ldots) As palavras-chave devem figurar logo abaixo do
 resumo, antecedidas da expressão Palavras-chave:, separadas entre si por
 ponto e finalizadas também por ponto.

 \textbf{Palavras-chave}: \textit{latex. abntex. editoração de texto}.
\end{resumo}

% resumo em inglês
\begin{resumo}[Abstract]

 \SingleSpacing{\citarautor~{\bfseries \rmfamily \fontsize{12}{12} \imprimirtitulo}. \imprimirdata. \thelastpage p. \tipotrabalhocurto~(\cursorealizado) -- \ppg, \imprimirinstituicao, \imprimirlocal.}

 \begin{otherlanguage*}{english}
   This is the english abstract.

   \vspace{\onelineskip}

   \noindent
   \textbf{Keywords}: \textit{latex. abntex. text editoration}.
 \end{otherlanguage*}
\end{resumo}

% resumo em francês
%\begin{resumo}[Résumé]

%  \SingleSpacing{\citarautor~{\bfseries \rmfamily \fontsize{12}{12} \imprimirtitulo}. \imprimirdata. \thelastpage p. \tipotrabalhocurto~(\cursorealizado) -- \ppg, \imprimirinstituicao, \imprimirlocal.}


% \begin{otherlanguage*}{french}
%    Il s'agit d'un résumé en français.
%
%   \textbf{Mots-clés}: latex. abntex. publication de textes.
% \end{otherlanguage*}
%\end{resumo}

% resumo em espanhol
%\begin{resumo}[Resumen]

%  \SingleSpacing{\citarautor~{\bfseries \rmfamily \fontsize{12}{12} \imprimirtitulo}. \imprimirdata. \thelastpage p. \tipotrabalhocurto~(\cursorealizado) -- \ppg, \imprimirinstituicao, \imprimirlocal.}


% \begin{otherlanguage*}{spanish}
%   Este es el resumen en español.
%
%   \textbf{Palabras clave}: latex. abntex. publicación de textos.
% \end{otherlanguage*}
%\end{resumo}
% ---

% ---
% inserir lista de ilustrações
% ---
\renewcommand{\listfigurename}{Lista de Figuras} % #modificado: Renomeia de Lista de Ilustrações para Lista de Figuras. Colocando em ppgec.cls não funcionou.
% ---
\pdfbookmark[0]{\listfigurename}{lof}
\listoffigures*
\cleardoublepage
% ---

% ---
% inserir lista de quadros
% ---
%\pdfbookmark[0]{\listofquadrosname}{loq}
%\listofquadros*
%\cleardoublepage
% ---

% ---
% inserir lista de tabelas
% ---
\pdfbookmark[0]{\listtablename}{lot}
\listoftables*
\cleardoublepage
% ---

% ---
% LISTA DE ABREVIATURAS E SIGLAS
% ---
\begin{siglas}
  \item[ABNT] Associação Brasileira de Normas Técnicas
  \item[abnTeX] ABsurdas Normas para TeX
\end{siglas}
% ---

% ---
% LISTA DE SÍMBOLOS
% ---
\begin{simbolos}
  \item[$ \Gamma $] Letra grega Gama
  \item[$ \Lambda $] Lambda
  \item[$ \zeta $] Letra grega minúscula zeta
  \item[$ \in $] Pertence
\end{simbolos}
% ---

% ---
% inserir o sumario
% ---
\pdfbookmark[0]{\contentsname}{toc}
\tableofcontents*
\cleardoublepage
% ---       % Escreva neste arquivo a DEDICATÓRIA, os AGRADECIMENTOS, a EPÍGRAFE, os RESUMOS, a LISTA DE ABREVIATURAS E SIGLAS e a LISTA DE SÍMBOLOS (pretextual.tex)
% ------------------------------------------------------------------------
% ------------------------------------------------------------------------

% ----------------------------------------------------------
% ELEMENTOS TEXTUAIS
% ----------------------------------------------------------
\textual
\pagestyle{fancy}
\fancyhf{}
\fancyhead[RO,LE]{\footnotesize\thepage}
\fancyfoot{}
\fancyfoot[CO]{\small{\imprimirautor~(\emailautor) \imprimirtipotrabalho. \abrevppg/\abrevinstituicao. \imprimirdata.}}
\fancyfoot[CE]{\small{\imprimirtitulo}}
\renewcommand{\headrulewidth}{0pt}
\renewcommand{\footrulewidth}{.4pt}
\captionsetup{width=0.75\textwidth} % incluido por Augusto B. Borges - padroniza largura da legenda das figuras e tabelas

% ----------------------------------------------------------
% ----------------------------------------------------------
\chapter{Introdução}\label{introducao}
% ----------------------------------------------------------

Este documento e seu código-fonte são exemplos de referência de uso da classe
\textsf{abntex2} e do pacote \textsf{abntex2cite}. O documento 
exemplifica a elaboração de tese ou dissertação no formato requerido pelo PPGEC-UFRGS, no qual seu formato não corresponde exatamente ao produzido conforme a ABNT NBR 14724:2011 \emph{Informação e documentação
- Trabalhos acadêmicos - Apresentação}.

Este modelo é baseado no \abnTeX\ mas modificado para o nosso Programa de Pós-Graduação e, futuramente, será colocado nos requisitos do \abnTeX. Uma lista completa das normas
observadas pelo \abnTeX\ original é apresentada em \citeonline{abntex2classe}.

Sinta-se convidado a participar do projeto original \abnTeX! Acesse o site do projeto em
\url{http://www.abntex.net.br/}. Também fique livre para conhecer,
estudar, alterar e redistribuir o trabalho do \abnTeX, desde que os arquivos
modificados tenham seus nomes alterados e que os créditos sejam dados aos
autores originais, nos termos da ``The \LaTeX\ Project Public
License''\footnote{\url{http://www.latex-project.org/lppl.txt}}. Isso permite que futuras versões do \abnTeX~não se tornem automaticamente
incompatíveis com as customizações promovidas. Consulte
\citeonline{abntex2-wiki-como-customizar} para mais informações.

Este documento deve ser utilizado como complemento dos manuais do \abnTeX\ 
\cite{abntex2classe,abntex2cite,abntex2cite-alf} e da classe \textsf{memoir}
\cite{memoir}. 

Espero, sinceramente, que o este modelo aprimore a qualidade do trabalho que
você produzirá, de modo que o principal esforço seja concentrado no principal:
na contribuição científica.


Augusto Bopsin Borges\footnote{modificada da mensagem original da Equipe \abnTeX~e de Lauro César Araujo}
  

             % Introdução
% ----------------------------------------------------------
% ----------------------------------------------------------
% %%%%%%%%%%%%%%%%%%%%%%%%%%%%%%%%%%%%%%%%%%%%%%%%%%%%%%%%%%%%%%%%%%%%%%%%%%%%%%%%%
%
%	cap#.tex 
%  -----------
%
%	Neste arquivo você escreve o capítulo. Deve sempre iniciar com o ambiente
%	\chapter{Nome do Capitulo}
%
% %%%%%%%%%%%%%%%%%%%%%%%%%%%%%%%%%%%%%%%%%%%%%%%%%%%%%%%%%%%%%%%%%%%%%%%%%%%%%%%%%
%
\chapter{Iniciando com o \LaTeX}

Nesse capítulo se dará um breve histórico e introdução sobre o que é o \TeX~e \LaTeX.

% ---
\section{O que é \TeX?}
% ---

O \TeX~é um programa de processamento de texto criado por Donald E. Knuth \cite{Knuth1984} a pedido da \textbf{AMS}\footnote{American Mathematical Society} orientado à composição, impressão de textos e fórmulas matemáticas. Esse programa interpreta cerca de 600 comandos que controlam a construção do layout de uma página (fonte de letra a usar, espaçamento entre linhas, organização de equações, entre outros aspectos referente ao texto final impresso). Para as fontes, Knuth aproveitou a experiência de antigos tipógrafos e desenvolveu o programa \texttt{METAFONT} para criá-las. Por isso, às vezes, percebe-se uma incrível semelhança entre livros antigos e os tipos de fontes utilizados pelo \TeX.

Pode se considerar o \TeX~ como sendo um compilador para textos científicos de alta qualidade. Contudo, sua aprendizagem e utilização não é tão fácil para qualquer usuário de computador. Felizmente, quase que simultaneamente foi desenvolvido o \LaTeX para simplificar a utilização do \TeX.

O desenvolvimento do \TeX~ se estendeu de 1977 a 1986, quando o mesmo foi posto de forma gratuita para ser utilizado. Atualmente o \TeX~ e o \texttt{METAFONT} não estão mais em desenvolvimentos e, segundo seu autor\footnote{Donald E. Kenuth. The Future of \TeX and Metafont. TUGboat, 11(4):489, novembro de 1990}, não serão realizados mais mudanças futuras, exceto correções de erros na programação.

% ---
\section{O que é \LaTeX?}
% ---

O \LaTeX, é um programa desenvolvido por Leslie Lamport \cite{1994lamport} que consiste em interpretar um conjunto de macros (que são instruções ou comandos) para simplificar o uso da linguagem \TeX. A utilização desses comandos permitem o autor compor e configurar seu documento de modo mais simples utilizando estruturas pré definidas.

Ao contrário do \TeX~ o desenvolvimento do \LaTeX~é crescente e possui diversos pacotes adicionais para realizar uma imensa quantidade de tarefas diferentes no processamento de textos.

O arquivo de entrada do \LaTeX~ está no formato ASCII e pode ser criado com qualquer editor de texto, como por exemplo o Bloco de Notas do Windows. Esse arquivo possuí tanto o texto que será impresso quanto as instruções que o \LaTeX~ interpretará para compor o texto no layout da página. Um projeto em \LaTeX~ pode conter diversos arquivos de entrada.

Existem vários editores de \LaTeX~ que servem de ambiente para escrever os arquivos de entrada. Esses programas possuem corretor ortográfico interativo, análise lexical (contagem de palavras e frases), realce de sintaxe, chamam e configuram o compilador, apresentam janelas de aviso e erros de compilação, permitem visualizar o documento impresso e possuem acesso rápido a comandos \LaTeX~ para fazer tabelas, fórmulas e etc. São exemplos de editores o TeXstudio, LyX, TeXmaker e OverLeaf. Para comparações entre diversos editores de \LaTeX~ veja \citeonline{wiki:2020}.

% ---
\section{Diferença entre processadores de texto Visuais e Lógicos}
% ---

Os programas que processam textos podem ser dividido em duas categorias:

\begin{alineas}
	
	\item Visuais (WYSIWYG): o texto que você digita aparece na tela da mesma forma que vai ser impresso. Isso é conhecido como WHAT YOU SEE IS WHAT YOU GET. Um exemplo é o Microsoft Word; 

	\item Lógico: o texto é digitado em um arquivo fonte junto com instruções de compilação. O arquivo é então compilado e gera uma saída que pode ser visualizada. Exemplo: HTML e \LaTeX.
	
\end{alineas}

Os processadores Visuais são bastante intuitivos e produzem documentos esteticamente bonitos. Porém pode ocorrer corrupção do arquivo, incompatibilidade na leitura entre versões diferentes do programa e problemas de configuração da formatação, uma vez que esta depende de diversos parâmetros do programa que podem variar sem muito controle do usuário.

Os processadores Lógicos, apesar de serem menos intuitivos para um usuário acostumado com o sistema Visual, possuem menos chances de corromper o arquivo e maior estabilidade na formatação, uma vez que é necessário indicar a estrutura lógica da mesma no texto. Além disso, pode-se citar algumas vantagens do \LaTeX:

\begin{alineas}
	
	\item é altamente portável e grátis. O sistema funciona em qualquer plataforma computacional;
		
	\item existem pacotes adicionais sem custo algum para muitas tarefas tipográficas que não estão inclusa no pacote do \LaTeX~básico;
	
	\item o usuário só precisa introduzir instruções simples para indicar a estrutura do documento e sua formatação. Os textos ficam bem estruturados no arquivo de entrada;

	\item estruturas como notas de rodapé, bibliografias, índices, tabelas e outras podem ser produzidas sem grande esforço;
	
	\item existem comunidades de dúvidas e desenvolvimento de pacotes por toda a internet. Uma dúvida que você tenha, certamente alguém já teve;
	
	\item é possível criar um conjunto de macros ou mesmo utilizar e customizar pacotes que contém uma série de macros criadas pela comunidade para configuração da formatação, como a classe Abn\TeX2 e o pacote abntex2cite;
	
\end{alineas}

Como uma das desvantagens pode-se citar que produzir um design específico de um documento que não esteja entre os pré-definidos pode ser uma tarefa de criação que leva tempo.

A mecânica do \LaTeX~ é aquela em que o autor escreve os arquivos de entrada *.tex (geralmente em um editor de \LaTeX) contendo o texto e os comandos de formatação na linguagem \LaTeX~ e então executa o compilador (um programa) que, seguindo as regras dessa linguagem, cria um arquivo de impressão formatado (*.pdf, *.dvi) pronto para ser visualizado.

% ---
\section{Instalando \LaTeX~ no sistema Windows}
% ---

Para utilizar o \LaTeX~no Windows você precisa primeiramente instalar o MiKTeX. O MiKTeX é uma distribuição multiplataforma que contém o compilador do \LaTeX, alguns pacotes básicos e outras ferramentas auxiliares (como um gerenciador de pacotes e um editor de \LaTeX~ chamado TeXworks). Posteriormente, é aconselhado instalar um editor de \LaTeX~ com mais funcionalidades, como o TeXstudio.

Para instalar o MiKTeX você deverá:

\begin{alineas}
	
	\item baixar o instalador do MiKTeX básico no site \url{https://miktex.org/download} correspondente ao seu sistema Windows 64-bit ou 32-bit;
	
	\item rode o instalador com privilégio de administrador;
	
	\item selecione instalar para todos os usuários e escolha a pasta. Recomenda-se manter o local sugerido;
	
	\item escolha as opções solicitadas, recomenda-se manter sugestão de folha \textbf{A4} para o tipo de papel e \textbf{Ask me First} para instalar os pacotes faltantes;
	
	\item clique em \textbf{start} para iniciar a instalação;
	
	\item após terminar feche o instalador;
	
\end{alineas}

Para instalar o TeXstudio você deverá:

\begin{alineas}
	
	\item baixar o instalador do TeXstudio no site \url{https://www.texstudio.org/} correspondente ao seu sistema Windows 64-bit ou 32-bit;
	
	\item rode o instalador com privilégio de administrador e escolha a língua de preferência;
	
	\item escolha a pasta de instalação. Recomenda-se manter o local sugerido;
	
	\item selecione criar ícones no desktop e na barra de acesso rápido;
	
	\item clique em \textbf{avançar} para iniciar a instalação;
	
	\item após terminar feche o instalador;
	
\end{alineas}

% ---
\section{Instalando \LaTeX~ em outros sistemas operacionais}
% ---

A instalação do \LaTeX~ em outros sistemas (Linux e Mac) segue passos equivalentes ao da seção anterior. Tanto o TeXstudio quanto o MikTeX possuem versões para essas outras plataformas. Contudo, verifique se todos os pacotes requeridos pelo arquivo ppgec.cls estão baixados. Caso não estejam, entre no MikTeX e baixe-os.

% ---
\section{Alguns tutoriais úteis}
% ---

Há muitos tutorias sobre \LaTeX~ disponíveis na internet. Abaixo segue a lista de alguns links que você pode achar útil:

\begin{alineas}
	
	\item curso básico sobre LaTeX no TeXstudio:\\ \url{https://youtube.com/playlist?list=PLTCY9jGgE91dFJwOP4wPvwLdyNj2Bch0D};
	
	\item curso básico ao avançado sobre OverLeaf:\\ \url{https://youtube.com/playlist?list=PLF6ZF9NW0WmqUAgtkYlQmCDHP6H_bYwGk};
	
	\item slides de um curso básico de \LaTeX~ do Alcemir Rodrigues Santos da UFBA:\\ \url{https://pt.slideshare.net/alcemirsantos/erbase-2015-curso-bsico-de-latex}
	
	\item site para formatar tabela e gerar código em \LaTeX:\\ \url{https://www.tablesgenerator.com/};
	
	\item site para gerar equações em \LaTeX:\\  \url{https://www.codecogs.com/latex/eqneditor.php?lang=pt-br};
	
	\item como obter uma bibliografia e colocar no seu trabalho:\\ \url{http://XXXXX};
	
\end{alineas}


% ---
\section{Sobre o modelo PPGEC, organização dos arquivos e utilização}
% ---
\begin{alineas}

	\item para baixar o modelo PPGEC: \\ \url{http://www.XXXXX}.

	\item organização dos arquivos e utilização do modelo PPGEC:\\ \url{https://XXXXX};

	\item modelo PPGEC com o Overleaf:\\ \url{https://www.youtube.com/watch?v=wK4mSMzB6KQ&t=184s};
	
\end{alineas}


             % Capítulo 2: Nome
% ----------------------------------------------------------
% ----------------------------------------------------------
\include{cap3}             % Capítulo 3: Nome
% ----------------------------------------------------------
% ----------------------------------------------------------
% ----------------------------------------------------------
\chapter{Novo Capítulo}
% ----------------------------------------------------------

% ---
\section{Oi, tudo bom?}
% ---

Minha terra tem palmeiras, onde canta o sabiá...
             % Capítulo 4: Nome
% ----------------------------------------------------------

% ----------------------------------------------------------
% ELEMENTOS PÓS-TEXTUAIS
% ----------------------------------------------------------
\postextual
% ----------------------------------------------------------
% Referências bibliográficas
% ----------------------------------------------------------
\bibliography{referencias}
% ----------------------------------------------------------
% Glossário
% ----------------------------------------------------------
%
% Consulte o manual da classe abntex2 para orientações sobre o glossário.
%
%\glossary

% ----------------------------------------------------------
% Apêndices
% ----------------------------------------------------------

% ---
% Inicia os apêndices
% ---
%\begin{apendicesenv}

% Imprime uma página indicando o início dos apêndices
%\partapendices

% ----------------------------------------------------------
%\chapter{Quisque libero justo}
% ----------------------------------------------------------

%\lipsum[50]

% ----------------------------------------------------------
%\chapter{Nullam elementum urna vel imperdiet sodales elit ipsum pharetra ligula
%ac pretium ante justo a nulla curabitur tristique arcu eu metus}
% ----------------------------------------------------------
%\lipsum[55-57]

%\end{apendicesenv}
% ---


% ----------------------------------------------------------
% Anexos
% ----------------------------------------------------------

% ---
% Inicia os anexos
% ---
%\begin{anexosenv}

% Imprime uma página indicando o início dos anexos
%\partanexos

% ---
%\chapter{Morbi ultrices rutrum lorem.}
% ---
%\lipsum[30]

% ---
%\chapter{Cras non urna sed feugiat cum sociis natoque penatibus et magnis dis
%parturient montes nascetur ridiculus mus}
% ---

%\lipsum[31]

% ---
%\chapter{Fusce facilisis lacinia dui}
% ---

%\lipsum[32]

%\end{anexosenv}

%---------------------------------------------------------------------
% INDICE REMISSIVO
%---------------------------------------------------------------------
% \phantompart
% \printindex
%---------------------------------------------------------------------

\end{document}
