% %%%%%%%%%%%%%%%%%%%%%%%%%%%%%%%%%%%%%%%%%%%%%%%%%%%%%%%%%%%%%%%%%%%%%%%%%%%%%%%%%
%
%	PACOTES_MACROS.tex 
%  --------------------
%
%	Este arquivo é lido no preâmbulo do main.tex e portanto, você pode inserir nele:
%
%		- pacotes que não estão na classe ppgec.cls (veja se a classe já não possui)
%		- suas próprias macros
%		- declarar operadores matemáticos e símbolos
%		- qualquer outro comando de preâmbulo
%
% %%%%%%%%%%%%%%%%%%%%%%%%%%%%%%%%%%%%%%%%%%%%%%%%%%%%%%%%%%%%%%%%%%%%%%%%%%%%%%%%%
%
%	Pacotes
%	-------
%
\usepackage{float}	% permite posicionar a figura entre parágrafos usando H
%
%
% %%%%%%%%%%%%%%%%%%%%%%%%%%%%%%%%%%%%%%%%%%%%%%%%%%%%%%%%%%%%%%%%%%%%%%%%%%%%%%%%%
%
%	Macros
%	------
%
% Imprimir anexo pdf: \imprimiranexopdf{nome do arquivo}{título do anexo}{primeirapagina}{proximaspaginas}
\newcommand{\imprimiranexopdf}[4]{
	\includepdf[pages=#3,pagecommand=\chapter{#2},scale=0.92,offset=0 -95]{#1}
	\includepdf[pages=#4,pagecommand={}]{#1}
}
%
%
% %%%%%%%%%%%%%%%%%%%%%%%%%%%%%%%%%%%%%%%%%%%%%%%%%%%%%%%%%%%%%%%%%%%%%%%%%%%%%%%%%
%
%	Declaração de operadores matemáticos e símbolos
%	-----------------------------------------------
%
% Alguns simbolos matemáticos de exemplo 
% (veja a lista de simbolos no arquivo simbolos.tex)
\DeclareMathOperator{\sen}{sen}
\DeclareMathOperator{\trace}{tr}
\DeclareMathOperator{\grad}{grad}
\DeclareMathOperator{\diverg}{div}
\DeclareMathOperator{\gravidade}{\texttt{g}}
\DeclareMathOperator{\vetorn}{\underline{\texttt{n}}}
\DeclareMathOperator{\desloc}{\underline{\xi}}
\DeclareMathOperator{\tensorsigma}{\underline{\underline{\sigma}}}
\DeclareMathOperator{\tensorSigma}{\underline{\underline{\Sigma}}}
\DeclareMathOperator{\tensorepsilon}{\underline{\underline{\varepsilon}}}
\DeclareMathOperator{\tensorBiot}{\underline{\underline{\textit{B}}}}
\DeclareMathOperator{\tensorpermeabilidademicro}{\underline{\underline{\textit{k}}}^{\prime}}
\DeclareMathOperator{\tensorpermeabilidadehom}{\underline{\underline{\textit{K}}}^{\prime \tiny{\mbox{hom}}}}
\DeclareMathOperator{\energiamedia}{\left\langle\textit{U}\right\rangle}
\DeclareMathOperator{\defmedia}{\langle\underline{\underline{\varepsilon}}\rangle}
\DeclareMathOperator{\tensaomedia}{\langle\underline{\underline{\sigma}}\rangle}
\DeclareMathOperator{\tensorC}{\mathbb{C}}
\DeclareMathOperator{\identidadequarta}{\mathbb{I}}
\DeclareMathOperator{\volume}{\left|\Omega\right|}
\DeclareMathOperator{\volumeinicial}{\left|\Omega_{0} \right|}
%
%
% %%%%%%%%%%%%%%%%%%%%%%%%%%%%%%%%%%%%%%%%%%%%%%%%%%%%%%%%%%%%%%%%%%%%%%%%%%%%%%%%%
%
%	Outros
%	------
%