% %%%%%%%%%%%%%%%%%%%%%%%%%%%%%%%%%%%%%%%%%%%%%%%%%%%%%%%%%%%%%%%%%%%%%%%%%%%%%%%%%
%
%	simbolo.tex 
%  -----------
%
%	Neste arquivo você escreve os simbolos do seu trabalho
%
%
% %%%%%%%%%%%%%%%%%%%%%%%%%%%%%%%%%%%%%%%%%%%%%%%%%%%%%%%%%%%%%%%%%%%%%%%%%%%%%%%%%
%
%	Criando simbolos com o \newcommand
%	-----------------------------------
%
%	Você pode criar comandos para simbolos para simplificar a notação matemática
%	ao longo do texto. Aqui tem alguns exmeplos:
%
\newcommand{\sen}{sen}
\newcommand{\trace}{tr}
\newcommand{\grad}{grad}
\newcommand{\diverg}{div}
\newcommand{\gravidade}{\texttt{g}}
\newcommand{\vetorn}{\underline{\texttt{n}}}
\newcommand{\desloc}{\underline{\xi}}
\newcommand{\tensorsigma}{\underline{\underline{\sigma}}}
\newcommand{\tensorSigma}{\underline{\underline{\Sigma}}}
\newcommand{\tensorepsilon}{\underline{\underline{\varepsilon}}}
\newcommand{\tensorBiot}{\underline{\underline{\textit{B}}}}
\newcommand{\tensorpermeabilidademicro}{\underline{\underline{\textit{k}}}^{\prime}}
\newcommand{\tensorpermeabilidadehom}{\underline{\underline{\textit{K}}}^{\prime \tiny{\mbox{hom}}}}
\newcommand{\energiamedia}{\left\langle\textit{U}\right\rangle}
\newcommand{\defmedia}{\langle\underline{\underline{\varepsilon}}\rangle}
\newcommand{\tensaomedia}{\langle\underline{\underline{\sigma}}\rangle}
\newcommand{\tensorC}{\mathbb{C}}
\newcommand{\identidadequarta}{\mathbb{I}}
\newcommand{\volume}{\left|\Omega\right|}
\newcommand{\volumeinicial}{\left|\Omega_{0} \right|}
%
% %%%%%%%%%%%%%%%%%%%%%%%%%%%%%%%%%%%%%%%%%%%%%%%%%%%%%%%%%%%%%%%%%%%%%%%%%%%%%%%%%
%
%	Colocando a lista de simbolos
%	------------------------------
%
%	A sintaxe para adicionar o simbolo na listagem é:
%	\item[simbolo] O que siginifica o simbolo
%
%
% Alguns exemplos de simbolos
\item[$ \Gamma $] 						Letra grega Gama
\item[$ \Lambda $] 						Lambda
\item[$ \zeta $] 						Letra grega minúscula zeta
\item[$ \in $] 							Pertence
%
% Simbolos definidos pelo usuário (ver seção acima)
\item[$ \sen $] 						Operador seno
\item[$ \trace $] 						Operador traço
\item[$ \grad $] 						Operador gradiente
\item[$ \diverg $]						Operador de divergência
\item[$ \gravidade $] 					Módulo da aceleração nominal da gravidade
\item[$ \vetorn $] 						Vetor n
\item[$ \desloc $] 						Vetor deslocamento
\item[$ \tensorsigma $] 				Tensor de tensões microscópico
\item[$ \tensorSigma $]					Tensor de tensões macroscópico
\item[$ \tensorepsilon $] 				Tensor de deformações microscópico
\item[$ \tensorBiot $] 					Tensor de Biot
\item[$ \tensorpermeabilidademicro $] 	Tensor de permeabilidade microscópico
\item[$ \tensorpermeabilidadehom $] 	Tensor de permeabilidade homogeneizado
\item[$ \energiamedia $] 				Energia Média
\item[$ \defmedia $] 					Deformação média
\item[$ \tensaomedia $] 				Tensão média
\item[$ \tensorC $] 					Tensor constitutivo de rigidez
\item[$ \identidadequarta $] 			Identidade de quarta ordem
\item[$ \volumeinicial $]				Volume inicial
\item[$ \volume $] 						Módulo de um \texttt{VER} considerado





